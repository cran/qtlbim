% \VignetteIndexEntry{QTLBIM Prototype Slides: User Customized Section}
% \VignetteDepends{qtlbim}
% \VignetteKeywords{QTL}
%\VignettePackage{qtlbim}



\section{User Customized Section}

\begin{frame}[fragile]
  \frametitle{Compare with Literature}

\tiny

Sugiyama et al. (2002) found:\\
two main QTLs on 1 4\\
two epistatic pairs with 6.15, 7.15

compare to present model:

\begin{Schunk}
\begin{Sinput}
> arch3 <- qb.arch(cross.step, main = c(1, 4), epistasis = data.frame(q1 = c(6, 
+     7), q2 = rep(15, 2)))
> arch3
\end{Sinput}
\end{Schunk}

\end{frame}

\begin{frame}[fragile]
  \frametitle{Sugiyama Model}

\tiny

\begin{Schunk}
\begin{Sinput}
> cross.step2 <- step.fitqtl(cross.sub, qtl, pheno.col, arch3)
\end{Sinput}
\end{Schunk}
\begin{Schunk}
\begin{Sinput}
> summary(cross.step2$fit)
\end{Sinput}
\end{Schunk}
\end{frame}


\begin{frame}[fragile]
  \frametitle{Sugiyama vs. Automata}

formal comparison with automated model

\tiny

\begin{Schunk}
\begin{Sinput}
> anova(cross.step, cross.step2)
\end{Sinput}
\end{Schunk}

\end{frame}

